\documentclass[11pt]{article}
\usepackage{fullpage}
\usepackage{graphicx}
\usepackage{longtable}
\usepackage{booktabs}
\usepackage{hyperref}

\providecommand{\tightlist}{%
  \setlength{\itemsep}{0pt}\setlength{\parskip}{0pt}}

\title{Requirements}
\date{}

\setlength\LTleft{0pt}
\setlength\LTright{0pt}

\setlength{\parskip}{\baselineskip}%
\setlength{\parindent}{0pt}%

\begin{document}
\maketitle
\tableofcontents
\newpage
\section{User Requirements}  

\begin{enumerate}
\item The program should check if an ant-brain supplied by a player is syntactically well-formed.
\item The program should check if a given description of an ant world is syntactically well-formed and meets the requirements for ant worlds used in tournaments.
\item The program should be able visualise a given ant world.
\item The program should be able to generate random but well-formed ant worlds.
\item The program should allow two players to play: i.e. enable two players to upload their ant-brains and choose an ant-world, and then run the game in the ant world, taking statistics and determining the winner of the game.
\item The program should allow users to play tournaments, where an arbitrary number of players can upload ant-brains, who are all paired up to play against each other. The overall tournament winner is the ant brain that wins the most individual games.
\item The development team should also produce a novel, proof-of-concept ant-brain.
\item The user can upload brains into an ant world with two ant colonies, one for each player.
\item The program should also support uploading custom ant world from files.
\item An ant world should simulate the behaviour of both kinds of ants, using the ant-brains supplied by the players.
\item The two species of ants are placed in a random world containing two anthills, some food sources, and several obstacles. 
\item Ants must explore the world, find food and bring it back to their anthill. 
\item Ants can communicate or leave trails by means of chemical markers.
\item Each species of ants can sense (with limited capabilities), but not modify, the markers of the other species.
\item Ants can also attack the ants of the other species by surrounding them
\item Ants that die as a result of an attack become food.
\item If an ant is adjacent to 5 (or 6) ants of the other species then it dies. However, if an ant is in a cell in the very corner of the grid or along one of the edges it can only have at most 4 ants adjacent to it - therefore it cannot be killed.
\item The match is won by the species with the most food in its anthill at the end of 300,000 rounds.
\item A cell can hold an unlimited amount of food during the course of the simulation.
\item In a tournament, each team will play every other team twice on each of the contest worlds. Once as red and once as black. So for 23 teams, each team will be pitted up against 22 other teams.
\item There must be at least one empty cell between adjacent non-food items.
\end{enumerate}

\section{Functional Requirements}

\subsection{Game}\label{game}

\subsubsection*{Game/1}\label{game1}

\begin{longtable}[c]{@{\extracolsep{\fill}}|p{0.2\textwidth}|p{0.7\textwidth}|@{}}
\hline
\textbf{Specification} & \textbf{Details}\tabularnewline
\hline
\textbf{Requirement} & Simulate a two-player game.\tabularnewline
\textbf{Description} & Using two ant-brains and an ant-world, carry out
a simulation of the game based on the ant finite-state machines and
their environment.\tabularnewline
\textbf{Inputs} & Two different colonies (consisting of a colour, a team
name and an ant-brain) and an ant-world.\tabularnewline
\textbf{Source} & The GUI event loop.\tabularnewline
\textbf{Outputs} & The final result of the game (which team won/lost).
The final state of the world.\tabularnewline
\textbf{Destination} & The GUI event loop.\tabularnewline
\textbf{Actions} & For each turn (maximum 300,000), the ants on each
cell shall be iterated over. Each ant will perform an action based on
their ant-brain (specified as a finite-state machine and graph). The
state of a cell (and possibly other ants) will be updated based on the
action of the ant. The winner of the game is determined by the colony
with the most food in their anthill (that is, the number of food
particles in their anthill's cells). Food carried by ants is ignored in
this score counting.\tabularnewline
\textbf{Notes} & -\tabularnewline
\hline
\end{longtable}

\subsubsection*{Game/2}\label{game2}

\begin{longtable}[c]{@{\extracolsep{\fill}}|p{0.2\textwidth}|p{0.7\textwidth}|@{}}
\hline
\textbf{Specification} & \textbf{Details}\tabularnewline
\hline
\textbf{Requirement} & Produce statistics for an ant game.\tabularnewline
\textbf{Description} & By keeping track of the state of the
world at all times (or having access to the history of the game), for
each team the following statistics will be computed: \begin{itemize}
 \item Amount of food in anthill
 \item Number of foe's ants killed
 \item Number of team's ants left
 \item Number of movements made
 \item Number of markings left
 \end{itemize} Further statistics can be added at the discretion of the programmers
where convenient.\tabularnewline
\textbf{Inputs} & The state of an ant-game at a particular turn (accumulative) or the full
history of an ant-game at the end of the game.\tabularnewline
\textbf{Source} & An ant game.\tabularnewline
\textbf{Outputs} & The statistics as specified above.\tabularnewline
\textbf{Destination} & The GUI statistics
display screen.\tabularnewline
\textbf{Actions} & Given the state of the game at a particular turn, all relevant statistics counters shall be updated. At the end of the game, the statistics shall be calculated and output. \tabularnewline
\textbf{Notes} & -\tabularnewline
\hline
\end{longtable}

\subsubsection*{Game/3}\label{game3}

\begin{longtable}[c]{@{\extracolsep{\fill}}|p{0.2\textwidth}|p{0.7\textwidth}|@{}}
\hline
Specification & Details\tabularnewline
\hline

\textbf{Requirement} & Simulate a tournament.\tabularnewline
\textbf{Description} & Plays ant games between multiple users, each with
their own ant-brain specification.\tabularnewline
\textbf{Inputs} & A list of teams (specified by a name and an
ant-brain), and an arbitrary (but at least 1) number of
ant-worlds.\tabularnewline
\textbf{Source} & The GUI.\tabularnewline
\textbf{Outputs} & The results of all matches, as well as the overall
winner (the team with the highest number of wins).\tabularnewline
\textbf{Destination} & The GUI tournament statistics display
screen.\tabularnewline
\textbf{Actions} & Iterates over all unique combinations
\texttt{(team-1,\ team-2,\ world)} and simulates a game with that
3-tuple, with the first team playing as red and the second team playing
as black. Keeps track of all wins, losses and statistics for each team.
After all simulations, (which are \emph{not} visualised to the user as
per \texttt{Game/1.3} but rather the statistics are displayed) the
overall winner is returned. The final results shall be calculated as
follows: a team shall earn 2 points for a win, and 1 point for a draw.
The score shall be tallied at the end, with the team having the highest
score being the winner. If there is a draw, the tournament shall be
repeated with the highest-scoring half of teams.\tabularnewline
\textbf{Notes} & Iterating over all unique 3-tuples as described above
will mean that every team will play every other team on every world as
both red and black. The number of worlds used for a tournament shall be
specified by the user when interacting with the GUI tournament setup
view.\tabularnewline
\hline
\end{longtable}

\subsubsection*{Game/4}\label{game4}

\begin{longtable}[c]{@{\extracolsep{\fill}}|p{0.2\textwidth}|p{0.7\textwidth}|@{}}
\hline
Specification & Details\tabularnewline
\hline

\textbf{Requirement} & Parse an ant brain file.\tabularnewline
\textbf{Description} & Checks if a user's custom ant brain (expected to
have been provided in a file on disk) is valid according to the ant
brain specification. If so, constructs an in-memory programmatic
representation of the brain.\tabularnewline
\textbf{Inputs} & A user-specified path to the ant-brain
file.\tabularnewline
\textbf{Source} & A file in-memory that has been read from
disk.\tabularnewline
\textbf{Outputs} & A programmatic representation of the ant
brain.\tabularnewline
\textbf{Destination} & The GUI event loop.\tabularnewline
\textbf{Actions} & Iterates through each line of the file, tokenising
the line and validating each token. Builds a graph of ant-brain
instructions which form a finite-state machine with transitions between
instructions dependent on the conditions specified in the
line.\tabularnewline
\textbf{Notes} & The specification of an ant-brain file is further
below: see \texttt{Parsing\ Specifications/Ant\ Brain}.\tabularnewline
\hline
\end{longtable}

\subsubsection*{Miscellaneous}\label{miscellaneous}

\begin{longtable}[c]{|p{0.2\textwidth}|p{0.5\textwidth}|p{0.2\textwidth}|}
\hline
Identifier & Requirement & Rationale\tabularnewline
\hline

\textbf{Game/Misc.1} & At the start of a game, all ants shall face east.
& User-requested.\tabularnewline
\textbf{Game/Misc.2} & Ant's brains shall be initialised to state 0. &
User requested.\tabularnewline
\textbf{Game/Misc.3} & Each anthill cell shall be initially populated
with a single ant of the anthill cell's colour. &
User-requested.\tabularnewline
\textbf{Game/Misc.4} & Ants shall be allocated identifiers in increasing
numerical order, in top-to-bottom left-to-right order based on their
position in the ant world. & User-requested.\tabularnewline
\hline
\end{longtable}

\subsection{Ant}\label{ant}

\subsubsection*{Ant/1}\label{ant1}

\begin{longtable}[c]{@{\extracolsep{\fill}}|p{0.2\textwidth}|p{0.7\textwidth}|@{}}
\hline
Specification & Details\tabularnewline
\hline

\textbf{Requirement} & Sense a cell.\tabularnewline
\textbf{Description} & Performs a sense operation on a particular cell
of a particular condition (cell and condition specified by the ant-brain
instruction).\tabularnewline
\textbf{Inputs} & Ant-brain instruction context, access to the cell in
question.\tabularnewline
\textbf{Source} & The game controller.\tabularnewline
\textbf{Outputs} & A boolean result that informs the game controller
which state the ant-brain has transitioned to (either the success state
or failure-state, as specified in the original ant-brain
file).\tabularnewline
\textbf{Destination} & The game control loop.\tabularnewline
\textbf{Actions} & The ant's current instruction stores information
about the direction to sense in and the condition to sense for. The
direction is one of \texttt{here}, \texttt{ahead}, \texttt{left-ahead}
or \texttt{right-ahead}. The ant queries the world to determine the cell
in the relevant direction, and checks the condition on that cell. If the
condition holds for that cell, it transitions to the \texttt{success}
state. If not, it transitions to the \texttt{fail}-state.\tabularnewline
\textbf{Notes} & The predicates available to an ant for sense operations
are \texttt{friend} (whether the given cell contains an ant of the same
team), \texttt{foe} (whether the given cell contains an ant of the
opposing team), \texttt{friend\_with\_food} (as with \texttt{friend},
but also carrying food), \texttt{foe\_with\_food} (as with \texttt{foe},
but also carrying food), \texttt{rock} (whether the cell type is rocky),
\texttt{marker\ n} (whether the cell is marked with a marker,
represented by an integer \texttt{n}, of the same team),
\texttt{foe\_marker} (whether the cell has marked by a foe, with
\emph{any} integer marker identifier), \texttt{home} (whether the cell
is part of the anthill of that ants team), and \texttt{foe\_home}
(whether the cell is part of the opponent's anthill).\tabularnewline
\hline
\end{longtable}

\subsubsection*{Ant/2}\label{ant2}

\begin{longtable}[c]{@{\extracolsep{\fill}}|p{0.2\textwidth}|p{0.7\textwidth}|@{}}
\hline
Specification & Details\tabularnewline
\hline

\textbf{Requirement} & Mark a cell.\tabularnewline
\textbf{Description} & Performs a mark operation on the current cell,
placing a chemical marker as specified by the current ant-brain
instruction.\tabularnewline
\textbf{Inputs} & The world context, the ant-brain instruction
context.\tabularnewline
\textbf{Source} & The game controller.\tabularnewline
\textbf{Outputs} & No output required.\tabularnewline
\textbf{Destination} & The game control loop.\tabularnewline
\textbf{Actions} & Tells the world to mark the ant's current cell with a
particular integer. Transitions to the next state.\tabularnewline
\textbf{Notes} & The marker must be in the range
\texttt{0..5}.\tabularnewline
\hline
\end{longtable}

\subsubsection*{Ant/3}\label{ant3}

\begin{longtable}[c]{@{\extracolsep{\fill}}|p{0.2\textwidth}|p{0.7\textwidth}|@{}}
\hline
Specification & Details\tabularnewline
\hline

\textbf{Requirement} & Unmark a cell.\tabularnewline
\textbf{Description} & Performs an unmark operation on the current cell,
removing a chemical marker as specified by the current ant-brain
instruction.\tabularnewline
\textbf{Inputs} & The world context, the ant-brain instruction
context.\tabularnewline
\textbf{Source} & The game controller.\tabularnewline
\textbf{Outputs} & No output required.\tabularnewline
\textbf{Destination} & The game control loop.\tabularnewline
\textbf{Actions} & Tells the world to unmark the ant's current cell,
removing the marker with a particular identifier. Transitions to the
next state.\tabularnewline
\textbf{Notes} & The marker must be in the range
\texttt{0..5}.\tabularnewline
\hline
\end{longtable}

\subsubsection*{Ant/4}\label{ant4}

\begin{longtable}[c]{@{\extracolsep{\fill}}|p{0.2\textwidth}|p{0.7\textwidth}|@{}}
\hline
Specification & Details\tabularnewline
\hline

\textbf{Requirement} & Pick up food.\tabularnewline
\textbf{Description} & Picks up food from the current cell,
transitioning to one of two ant-brain states depending on whether the
pick-up operation was successful.\tabularnewline
\textbf{Inputs} & The world context, the ant-brain instruction
context.\tabularnewline
\textbf{Source} & The game controller.\tabularnewline
\textbf{Outputs} & A boolean result that informs the game controller
which state the ant-brain has transitioned to (either the success state
or failure-state, as specified in the original ant-brain
file).\tabularnewline
\textbf{Destination} & The game control loop.\tabularnewline
\textbf{Actions} & Queries the world to see if the current cell has
food. If not, transitions to the fail-state. Otherwise ets the internal
state of the ant to \texttt{has-food\ =\ true}, as per requirement
\texttt{Ant/Misc.4}. Removes a food particle from the current cell.
Transitions to the success state.\tabularnewline
\textbf{Notes} & -\tabularnewline
\hline
\end{longtable}

\subsubsection*{Ant/5}\label{ant5}

\begin{longtable}[c]{@{\extracolsep{\fill}}|p{0.2\textwidth}|p{0.7\textwidth}|@{}}
\hline
Specification & Details\tabularnewline
\hline

\textbf{Requirement} & Drop food.\tabularnewline
\textbf{Description} & Drops food into the current cell, transitioning
to the next state as specified in the ant-brain finite state
machine.\tabularnewline
\textbf{Inputs} & The world context, the ant-brain instruction
context.\tabularnewline
\textbf{Source} & The game controller.\tabularnewline
\textbf{Outputs} & No output required.\tabularnewline
\textbf{Destination} & The game control loop.\tabularnewline
\textbf{Actions} & Gets the current cell from the world context and adds
a food particle to it. Transitions to the next state.\tabularnewline
\textbf{Notes} & If the ant does not have food, no action on the cell
shall be taken, and the ant shall transition to the next
state.\tabularnewline
\hline
\end{longtable}

\subsubsection*{Ant/6}\label{ant6}

\begin{longtable}[c]{@{\extracolsep{\fill}}|p{0.2\textwidth}|p{0.7\textwidth}|@{}}
\hline
Specification & Details\tabularnewline
\hline

\textbf{Requirement} & Turn.\tabularnewline
\textbf{Description} & Causes the internal direction attribute of the
ant (as specified in \texttt{Ant/Misc.7}) to change based on the current
turn instruction's specification.\tabularnewline
\textbf{Inputs} & The world context, the ant-brain instruction
context.\tabularnewline
\textbf{Source} & The game controller.\tabularnewline
\textbf{Outputs} & No output required.\tabularnewline
\textbf{Destination} & The game control loop.\tabularnewline
\textbf{Actions} & The direction specified by the instruction can be one
of \texttt{left} or \texttt{right}. This shall cause the direction of
the ant to `rotate' - i.e. \texttt{left} will cause the ant to rotate
one edge anticlockwise in its hexagonal cell, and vice-versa for
\texttt{right}. The ant will then transition into the next
state.\tabularnewline
\textbf{Notes} & -\tabularnewline
\hline
\end{longtable}

\subsubsection*{Ant/7}\label{ant7}

\begin{longtable}[c]{@{\extracolsep{\fill}}|p{0.2\textwidth}|p{0.7\textwidth}|@{}}
\hline
Specification & Details\tabularnewline
\hline

\textbf{Requirement} & Move.\tabularnewline
\textbf{Description} & Causes the ant to move forward based on its
direction into an adjacent cell (if possible).\tabularnewline
\textbf{Inputs} & The world context, the ant-brain instruction
context.\tabularnewline
\textbf{Source} & The game controller.\tabularnewline
\textbf{Outputs} & A boolean result that informs the game controller
which state the ant-brain has transitioned to (either the success state
or failure-state, as specified in the original ant-brain
file).\tabularnewline
\textbf{Destination} & The game control loop.\tabularnewline
\textbf{Actions} & Based on the ant's current direction and the ant's
current cell, the world can be queried for the cell in front of the ant.
If the cell is free (not rocky and does not contain another ant), the
ant shall remove itself from the current cell and add itself to the
relevant adjacent cell. It shall then update its own \texttt{resting}
attribute to indicate to the game controller that it must now rest for
14 moves. Then it will transition to the success state. If not free, the
ant shall transition to the fail-state.\tabularnewline
\textbf{Notes} & Resting is not \emph{handled} here, rather it is
handled by the game controller which will check the resting attribute
and not call any instruction-step methods on the ant unless not
resting.\tabularnewline
\hline
\end{longtable}

\subsubsection*{Ant/8}\label{ant8}

\begin{longtable}[c]{@{\extracolsep{\fill}}|p{0.2\textwidth}|p{0.7\textwidth}|@{}}
\hline
Specification & Details\tabularnewline
\hline

\textbf{Requirement} & Flip.\tabularnewline
\textbf{Description} & Causes the ant to transition into one of two
states depending on the result of a pseudo-random generated
integer.\tabularnewline
\textbf{Inputs} & The world context, the ant-brain instruction
context.\tabularnewline
\textbf{Source} & The game controller.\tabularnewline
\textbf{Outputs} & A boolean result that informs the game controller
which state the ant-brain has transitioned to (either state-1 or
state-2, as specified in the original ant-brain file).\tabularnewline
\textbf{Destination} & The game control loop.\tabularnewline
\textbf{Actions} & Using the system's random number generator, the ant
shall produce a random integer between \texttt{0..(n-1)}, where
\texttt{n} is the integer specified by the flip instruction. If
\texttt{0}, the ant shall transition to state-1. Otherwise, it shall
transition to state-2.\tabularnewline
\textbf{Notes} & -\tabularnewline
\hline
\end{longtable}

\subsubsection*{Miscellaneous}\label{miscellaneous-1}

\begin{longtable}[c]{|p{0.2\textwidth}|p{0.3\textwidth}|p{0.4\textwidth}|}
\hline
Identifier & Requirement & Rationale\tabularnewline
\hline

\textbf{Ant/Misc.1} & Each ant shall have an associated unique integer
identifier. & This identifier shall determine the order in which the ant
moves as compared to other ants (with ants moving in ascending order of
identifier).\tabularnewline
\textbf{Ant/Misc.2} & Each ant shall have an associated colour, either
\texttt{red} or \texttt{black}. & The colour shall identify the ant's
team and also make it clear to the user in world
visualisation.\tabularnewline
\textbf{Ant/Misc.3} & Each ant shall use the finite-state machine
representation of its brain to store its current state. & This state is
queried by the game controller to decide how the ant interacts with the
world.\tabularnewline
\textbf{Ant/Misc.4} & Each ant shall hold a \texttt{boolean} field that
tracks whether the ant is holding food. & Accessible to the game
controller since it performs different actions depending on whether an
ant carries food or not.\tabularnewline
\textbf{Ant/Misc.5} & Each ant shall hold an \texttt{integer} field that
tracks how many turns it is resting for. & When an ant rests, it is not
allowed to move. This field shall be decremented by the game loop, and
if \texttt{0} then the ant is able to move.\tabularnewline
\textbf{Ant/Misc.6} & Ants shall die if surrounded by \texttt{5} ants of
the opposing team. & User-specified requirement. Note: this means that
ants on the edge of the game-world or next to certain rocky structures
are not able to be killed, since they do not have \texttt{5} free cells
surrounding them.\tabularnewline
\textbf{Ant/Misc.7} & The ant shall track its own direction. & The
direction of the ant shall be used by the game controller for operations
such as moving the ant forward or sensing.\tabularnewline
\hline
\end{longtable}

\subsection{World}\label{world}

\subsubsection*{World/1}\label{world1}

\begin{longtable}[c]{@{\extracolsep{\fill}}|p{0.2\textwidth}|p{0.7\textwidth}|@{}}
\hline
Specification & Details\tabularnewline
\hline

\textbf{Requirement} & Parse ant world file.\tabularnewline
\textbf{Description} & Checks if a user's custom ant world (expected to
have been provided in a file on disk) is valid according to the ant
world specification. If so, constructs an in-memory programmatic
representation of the world.\tabularnewline
\textbf{Inputs} & A user-specified path to the ant-world
file.\tabularnewline
\textbf{Source} & A file in-memory that has been read from
disk.\tabularnewline
\textbf{Outputs} & A programmatic representation of the ant
world.\tabularnewline
\textbf{Destination} & The GUI event loop.\tabularnewline
\textbf{Actions} & Iterates through each line of the file, tokenising
the line and validating each token. Intepreting each token as a cell and
building a hexagonal programmatic structure containing cells with
properties specified in the file.\tabularnewline
\textbf{Notes} & -\tabularnewline
\hline
\end{longtable}

\subsubsection*{World/2}\label{world2}

\begin{longtable}[c]{@{\extracolsep{\fill}}|p{0.2\textwidth}|p{0.7\textwidth}|@{}}
\hline
Specification & Details\tabularnewline
\hline

\textbf{Requirement} & Generate random contest worlds.\tabularnewline
\textbf{Description} & Creates random contest worlds that conform to the
requirements for a tournament, as described in the \emph{Actions}
section below.\tabularnewline
\textbf{Inputs} & None.\tabularnewline
\textbf{Source} & The tournament setup, or alternatively the GUI. That
is, this has two different operations - either the user requests a
custom ant world (to write to file), or the contest world is randomly
generated as a tournament is starting.\tabularnewline
\textbf{Outputs} & A programmatic representation of an ant
world.\tabularnewline
\textbf{Destination} & The GUI event loop (or tournament
setup).\tabularnewline
\textbf{Actions} & Pseudo-randomly generates an ant world of 150x150
cells. This ant world shall have rocky edge cells. It shall have two
hexagonal ant hills of edge length 7, for red and black, randomly
located. It shall have 14 randomly located rocks. It shall have 11
randomly located blobs of food, where a blob of food is a 5x5 rectangle.
Each food blob shall be randomly oriented and each cell in the ant blob
shall have 5 food particles. All separate components of the world
(ant-hills, food-blobs, rocks) shall be separated by at least one clear
cell.\tabularnewline
\textbf{Notes} & A component for converting a programmatic
representation of an ant world to a textual representation should also
be developed.\tabularnewline
\hline
\end{longtable}

\subsubsection*{Miscellaneous}\label{miscellaneous-2}

\begin{longtable}[c]{|p{0.2\textwidth}|p{0.3\textwidth}|p{0.4\textwidth}|}
\hline
Identifier & Requirement & Rationale\tabularnewline
\hline

\textbf{World/Misc.1} & The ant world programmatic structure shall
present an interface that appears as if the world is comprised of
hexagonal cell (that is, each non-edge cell should be surrounded by 6
adjacent cells). & As specified in the user requirements, each cell is a
hexagon.\tabularnewline
\textbf{World/Misc.2} & Each cell shall have an associated type, which
shall be one of \texttt{rocky}, \texttt{clear}, \texttt{red-ant-hill} or
\texttt{black-ant-hill}. & As per user requirements, \texttt{rocky}
cells are inaccessible to ants, and \texttt{ant-hill} cells belong to
their associated teams and are where ants are initially
spawned.\tabularnewline
\textbf{World/Misc.4} & Ant hills cannot be adjacent to one another. &
Direct user requirement, reasoning unclear.\tabularnewline
\textbf{World/Misc.5} & Each non-rocky cell can contain \texttt{0-1}
ants. & Direct user requirement. Ants cannot `walk over' each
other.\tabularnewline
\textbf{World/Misc.6} & Each clear cell can contain \texttt{0+} food
particles. & Food cannot logically be negative. Note: as specified by
the ant-world cell specification, initially a cell can contain only
\texttt{0-9} food particles. However, as food is dropped this number can
increase above the initial limit unboundedly.\tabularnewline
\textbf{World/Misc.7} & Each clear cell can contain any non-negative
number of chemical markers for each team. & Ants can mark the same cell
with multiple markers, but cannot modify their opponents.\tabularnewline
\hline
\end{longtable}

\subsection{GUI}\label{gui}

\subsubsection*{GUI/1}\label{gui1}

\begin{longtable}[c]{@{\extracolsep{\fill}}|p{0.2\textwidth}|p{0.7\textwidth}|@{}}
\hline
Specification & Details\tabularnewline
\hline

\textbf{Requirement} & Visualise ant world.\tabularnewline
\textbf{Description} & Uses a Swing GUI to show users a representation
of the ant world in a game at a particular turn.\tabularnewline
\textbf{Inputs} & An ant world containing information about cells,
including their location, their type, and what they contain (e.g.~ants,
food).\tabularnewline
\textbf{Source} & A currently-running ant game.\tabularnewline
\textbf{Outputs} & An image (or otherwise) showing the state of the game
in a human-readable fashion, displayed using a Swing
interface.\tabularnewline
\textbf{Destination} & The game loop.\tabularnewline
\textbf{Actions} & Loops through all cells in the game and draws them to
screen, with an indication of the type of cell and what it contains
specified by a colour (colours not yet specified, will be at the
discretion of the programmers and testers).\tabularnewline
\textbf{Notes} & -\tabularnewline
\hline
\end{longtable}

\subsubsection*{Miscellaneous}\label{miscellaneous-3}

\begin{longtable}[c]{|p{0.2\textwidth}|p{0.3\textwidth}|p{0.4\textwidth}|}
\hline
Identifier & Requirement & Rationale\tabularnewline
\hline

\textbf{GUI/Misc.1} & The GUI shall display the entire world, including
all world content. & Allows viewing of the entire state of the game in a
user-friendly way.\tabularnewline
\textbf{GUI/Misc.2} & The game view interface shall scale proportionally
to world dimensions \texttt{(x,\ y)}. & Scaling means the entire world
state is always visible to users.\tabularnewline
\textbf{GUI/Misc.3} & There will be a user-customisable option to
display the state of the game every \texttt{n} frames. & Makes the game
play-speed personalisable to the user, to avoid simulation being too
short or too long.\tabularnewline
\hline
\end{longtable}

\subsection{Parsing Specifications}\label{parsing-specifications}

\subsubsection{Brain}\label{ant-brain}

This is the specification for an ant-brain file. These are provided by
the user, but development of the brain parser shall interpret the file
contents as conforming to the following (taken directly from the client
requirements):

\begin{itemize}
\tightlist
\item
  Each line in the file represents one state. The first line is state 0,
  the second line state 1, and so on.
\item
  The file may contain at most 10000 lines.
\item
  Each line shall consists of a sequence of whitespace-separated tokens,
  followed (optionally) by a comment beginning with a semicolon and
  extending to the end of the line.
\item
  Tokens are either keywords or integers.
\item
  Keywords are case-insensitive.
\item
  The possible instruction tokens are:
  \begin{itemize}
  \tightlist
  \item
  \texttt{Sense}
\item
  \texttt{Mark}
\item
  \texttt{Unmark}
\item
  \texttt{PickUp}
\item
  \texttt{Drop}
\item
  \texttt{Turn}
\item
  \texttt{Move}
\item
  \texttt{Flip}
  \end{itemize}
\item
  The tokens for sensing directions are:
  \begin{itemize}\tightlist
  \item
  \texttt{Here}
\item
  \texttt{Ahead}
\item
  \texttt{LeftAhead}
\item
  \texttt{RightAhead}
  \end{itemize}
\item
  The tokens for conditions are:
    \begin{itemize}\tightlist
\item
  \texttt{Friend}
\item
  \texttt{Foe}
\item
  \texttt{FriendWithFood}
\item
  \texttt{FoeWithFood}
\item
  \texttt{Food}
\item
  \texttt{Rock}
\item
  \texttt{Marker\ i}, where \texttt{i} is an integer from \texttt{0-5}
\item
  \texttt{FoeMarker}
\item
  \texttt{Home}
\item
  \texttt{FoeHome}
  \end{itemize}
\item
  The tokens for turn are:
    \begin{itemize} \tightlist
\item
  \texttt{Left}
\item
  \texttt{Right}
  \end{itemize}
\end{itemize}

\subsubsection{World}\label{ant-world}

This is the specification for an ant-world file. These are both provided
by the user and generated randomly, so development of the world parser
and contest world generator shall conform to the following (taken
directly from the client requirements):
\begin{itemize}
\tightlist
\item
  The first line of the file contains a single integer representing the
  size of the world in the \texttt{x} dimension.
\item
  The second line of the file contains a single integer representing the
  size of the world in the \texttt{y} dimension.
\item
  The rest of the file consists of \texttt{y} lines, each containing
  \texttt{x} one-character cell specifiers, separated by spaces (even
  lines also contain a leading space before the first cell specifier).
  The top-left cell specifier corresponds to position \texttt{(0,\ 0)}.
  \textbf{Note}: although generated worlds shall conform to this
  specification, the parser shall ignore leading and extra whitespace.
\item
  The cell specifiers are:
  \begin{itemize} \tightlist
\item
  \texttt{\#}: rocky cell
\item
  \texttt{.}: clear cell (containing nothing interesting)
\item
  \texttt{+}: red anthill cell
\item
  \texttt{-}: black anthill cell
\item
  \texttt{1}-\texttt{9}: clear cell containing the given number of food
  particles
  \end{itemize}
\end{itemize}

\newpage
\section{Non-Functional Requirements}

\subsubsection{Product}\label{product}

\subparagraph{Reliability}\label{reliability}

\begin{enumerate}
\def\labelenumi{\arabic{enumi}.}
\tightlist
\item
  The software should crash no more than 1 time in 1000 when parsing
  user files, even if the files are corrupt or not well-formed.
\item
  The software should crash no more than 1 time in 1000 when simulating
  a game or tournament.
\item
  The GUI should be stable and responsive to user input at least 99\% of
  the time, including when a game is running. This applies only to
  systems with more than 400MB of free RAM.
\item
  The software shall parse user files of up to 10,000 lines each without
  crashing more than 1 time in 1000.
\item
  Visualisations of games and tournaments should proceed with no visible
  defects (glitches) at least 99 times out of 100.
\item
  If the program crashes, a log containing the stack trace of the error
  and any other pertinent details shall be written to the program's
  working directory, titled ``ant-game-crash-{[}current unix
  time{]}.log''.
\end{enumerate}

\subparagraph{Usability}\label{usability}

\begin{enumerate}
\def\labelenumi{\arabic{enumi}.}
\tightlist
\item
  The software should have a minimal learning curve for new users. That
  is, the GUI should have intuitive and clear controls that step the
  user(s) through each stage of the game without being required to
  follow external instructions (not including understanding the rules of
  the game and the development of a user's ant-brain file).
\item
  All user interfaces shall be developed using the JavaFX API.
\item
  The software shall present a help menu, accessible at all times in the
  program, which upon clicking displays a popup user interface
  containing the searchable user documentation for the program.
\end{enumerate}

\subparagraph{Performance}\label{performance}

\begin{enumerate}
\def\labelenumi{\arabic{enumi}.}
\tightlist
\item
  Response time of the system when interacting with any component of the
  GUI should be negligible before the user is presented with the
  interaction's result or with a loading screen.
\item
  The GUI should not hang (that is, be unresponsive to user interaction)
  for more than one second after any user interaction.
\item
  Parsing user files (ant-brain, ant-world) should take no longer than 5
  seconds.
\item
  The two-player game simulation should take no longer than 30 seconds
  for 300,000 rounds, not including artificially introduced delays when
  displaying the game to the user.
\end{enumerate}

\subparagraph{Security}\label{security}

\begin{enumerate}
\def\labelenumi{\arabic{enumi}.}
\tightlist
\item
  The software shall not require internet or network access to run, and
  will have access to the full feature-set of the program offline
  without use of any network connections.
\item
  The system shall not purposefully access or alter any existing user or
  system files, with the exception of ant-brain and ant-world files
  chosen with explicit permission from the user (via selection from a
  file-chooser interface).
\end{enumerate}

\subparagraph{Portability}\label{portability}

\begin{enumerate}
\def\labelenumi{\arabic{enumi}.}
\tightlist
\item
  The software shall be cross-platform across commonly used personal
  desktop operating systems (Windows 7+, Mac OS X 10.8.3+, Linux) that
  support Java 8.0.
\end{enumerate}

\subparagraph{Resources}\label{resources}

\begin{enumerate}
\def\labelenumi{\arabic{enumi}.}
\tightlist
\item
  The system should not store more than 100MB of non-volatile data, not
  including the user's ant-brain and ant-world files.
\item
  The software shall use no more than 100MB of RAM, with the exception
  of a maximum of 400MB of RAM while executing/displaying a game.
\end{enumerate}

\subparagraph{Deployment}\label{deployment}

\begin{enumerate}
\def\labelenumi{\arabic{enumi}.}
\tightlist
\item
  A build of the program shall be delivered in executable JAR format, to
  allow the user to run the program without accessing the command line
  and without any other external configuration by the user.
\end{enumerate}

\subsubsection{Organisation}\label{organisation}

\begin{enumerate}
\def\labelenumi{\arabic{enumi}.}
\tightlist
\item
  All development and testing shall be undertaken in either the Netbeans
  IDE or IntelliJ IDE.
\item
  Team communication relevant to the product-related aspects of the
  project (planning, development, testing) shall be recorded for future
  reference:

  \begin{enumerate}
  \def\labelenumii{\arabic{enumii}.}
  \tightlist
  \item
    Facebook chat - message archive
  \item
    Slack chat - message archive
  \item
    In-person - meeting minutes
  \end{enumerate}
\item
  JUnit 4.12 shall be used as the testing library across all development
  and testing systems.
\item
  Gradle 2.5+ shall be used as the build automaton system across all
  development and testing systems.
\item
  All project-related documents, source code, tests and builds shall be
  kept on the team's GitHub repository, with new versions being
  committed and pushed to the repository on a regular basis and as soon
  as a major change/refactoring occurs.
\item
  The software's source code shall be documented extensively, defined as
  follows:

  \begin{enumerate}
  \def\labelenumii{\arabic{enumii}.}
  \tightlist
  \item
    The
    \href{http://www.oracle.com/technetwork/java/javase/documentation/index-jsp-135444.html}{Javadoc
    comment structure} shall be used to document every field, method and
    class.
  \item
    The structure of Javadoc documentation shall follow the
    \href{http://www.oracle.com/technetwork/java/javase/documentation/index-137868.html}{Oracle
    style guide}.
  \item
    Single-line comments shall be used inside methods to make the
    purpose of lines/blocks of code clear, when deemed appropriate by
    the coder(s).
  \item
    For fragments of incomplete code which are pushed to the project's
    GitHub repository or left to complete later, a single line comment
    starting with ``//TODO:'' shall be used to briefly describe the task
    to complete.
  \end{enumerate}
\item
  A copy of all project documents and source code shall be stored
  independently on each team member's personal computer, regularly
  pulled from the central team GitHub repository. In addition, an
  up-to-date backup of the GitHub repository shall be kept in one or
  more of the team's personal Dropbox repositories.
\item
  The Jacoco code coverage analysis tool shall be used by the
  testing/validation team to ensure maximum code coverage. The final
  submitted system shall have 100\% code coverage.
\end{enumerate}

\subsubsection{External}\label{external}

\begin{enumerate}
\def\labelenumi{\arabic{enumi}.}
\tightlist
\item
  The software shall be delivered by Thursday 5 May at 4:00PM, by online
  submission to Study Direct.
\item
  All libraries and external material used in the project shall be
  documented and credited to the original author(s), as per the
  University of Sussex's
  \href{http://www.sussex.ac.uk/s3/?id=35}{plagiarism} policies.
\end{enumerate}

\newpage
\section{Domain Requirements}

There are no additional domain requirements for this project. The application domain of the system is assumed to be extremely similar to the development and testing domain: as per the non-functional product requirements, this software is designed to run on personal operating systems compatible with Java 8.

All current requirements have been classified as functional or non-functional.

\end{document}

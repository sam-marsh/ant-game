\documentclass[11pt]{article}
\usepackage{fullpage}
\usepackage{graphicx}
\usepackage{longtable}
\usepackage{booktabs}
\usepackage{hyperref}

\providecommand{\tightlist}{%
  \setlength{\itemsep}{0pt}\setlength{\parskip}{0pt}}

\title{Quality Assurance}
\date{}

\setlength\LTleft{0pt}
\setlength\LTright{0pt}

\setlength{\parskip}{\baselineskip}%
\setlength{\parindent}{0pt}%

\begin{document}
\maketitle
\tableofcontents
\newpage

\section{Test Specification}\label{test-specification}

\subsection{Scope}\label{scope}

The tests will cover all possible control flow in the program. This will
be examined by \href{http://eclemma.org/jacoco/}{JaCoCo}, a code
coverage library for Java. JaCoCo examines the bytecode of the program,
and produces percentages of the statements covered by existing tests for
each method, each class, and overall.

Tests shall aim to keep code coverage above 80\% at all times. As
discussed in the project plan, this will be managed by splitting the
team into two groups for the development stage - one group to write
code, and the other to write tests for it.

In addition, code coverage should aim to be `above' 100\% in areas:
although not required, the members of the testing group will aim to
assist the developers using test-driven development - i.e.~by writing
tests for a particular program submodule before the coders have built
it.

Tests shall be primarily aimed at targeting the explicit functional
requirements of the project. As such, each functional requirement shall
generate multiple tests, aiming to strictly enforce that functional
requirement.

Tests shall also enforce program behaviours described by the class
diagrams and sequence diagrams of the modelling phase.

The continuous integration software used in the remote Git repository,
Travis CI, runs code coverage and automatics tests upon pushes to the
repository (regression testing). The status of these operations,
displayed in \texttt{README.md} in the root project directory, shall be
monitored to ensure test coverage is up-to-date and that code is in line
with the requirements.

\begin{center}
\includegraphics{build-passing} \includegraphics{code-coverage}
\end{center}

When a test fails, the programmers shall be alerted to the location of
the error, pertinent details, and the related functional requirement.
This will allow the developers to adjust the program to suit the user's
requirements. Where the source of an error cannot be found, breakpoints
shall be used for debugging.

Tests shall have extensive documentation. Unit test classes shall
contain class-level documentation including a link to the class being
tested. For example:

\texttt{This\ class\ is\ testing:\ \{@link\ antgame.core.Ant\}}

In addition, each test method shall contain links to each method that is
being tested. It shall also describe the functional requirement that the
test is written to validate. For example:

\texttt{This\ method\ tests\ that\ an\ ant\ is\ not\ able\ to\ move\ onto\ an\ already-occupied\ cell:\ \{@link\ antgame.core.Ant\#move(int)\}}

\subsection{Test Results}

Final test results of the submission are included in a \texttt{.zip} folder alongside this document. The test results are presented in HTML format. 

\subsection{Test Plan}\label{test-plan}

\subsubsection{Test Phases}\label{test-phases}

\subparagraph{Phase 1 - Unit Testing}\label{phase-1---unit-testing}

This phase is to be worked on concurrently while the programming
sub-team produces code. As described above, code coverage should be
maintained at above 80\% at all times.

JUnit 4.2 shall be used as the testing suite on all the tester's
platforms.

\subparagraph{Phase 2 - Release
Testing}\label{phase-2---release-testing}

This phase takes place after development is complete. As per the project
plan, this should take approximately one week to complete thoroughly.

Release testing shall be undertaken by the entire team.

It shall involve running the entire system, interacting through the GUI,
and examining the system's response to both typical and edge cases
(e.g.~attempting to load an empty ant brain, or adding no teams to a
tournament).

\subparagraph{Overhead
Software/Libraries}\label{overhead-softwarelibraries}
\begin{itemize}
\tightlist
\item
  JUnit 4.2 - external library, managed with Gradle
\item
  JaCoCo - external library, managed with Gradle
\end{itemize}

\newpage
\subsection{Unit Tests}\label{unit-tests}

\subsubsection{Brain}\label{brain}

\begin{longtable}[c]{@{}p{0.2\textwidth}p{0.7\textwidth}@{}}
\toprule
& Instruction Tests \tabularnewline
\midrule
Prerequisites & - \tabularnewline
Description & We shall be testing the functionality of the \texttt{Sense}, \texttt{Mark}, \texttt{Unmark}, \texttt{Drop}, \texttt{Pickup}, \texttt{Move}, \texttt{Flip} and \texttt{Turn} instructions followed by the
\texttt{Condition} class representing a type of condition used by an \texttt{Ant} during a sense. The Condition will be tested by ensuring all conditions match their type and \texttt{getMarker()} on a non marker condition followed by a marker condition. Sense will be tested by ensuring whether the direction and condition of the sense are set correctly. The \texttt{Mark} and \texttt{Unmark} instruction's \texttt{getMarker()} will be tested to ensure the marker to place/remove is returned correctly. \texttt{Flip} will be tested to ensure the upper bound on the random number generator is set correctly. \texttt{Drop}, \texttt{Pickup} and \texttt{Move} will be tested to ensure they have the correct identifier and
\texttt{Instruction.Type}, and transition to the next state appropriately on success and failure. \tabularnewline
Expected Result & A successful test is determined by whether an
instruction has the correct Instruction identifer and that their type matches
the tested instruction and whether it transitions to the next state on
success and failure appropriately. In terms of the \texttt{Condition}, all
conditions should match their type and \texttt{getMarker()} on a marker condition
should return a marker and on a non marker condition should throw an
exception. \texttt{Sense} should set the direction and condition of the
instruction correctly. The \texttt{Mark} and \texttt{Unmark} instructions \texttt{getMarker()}
method should return an instruction. The \texttt{Flip} instructions's \texttt{getRange()}
should return the upper bound on the random number generated. Finally
\texttt{Drop}, \texttt{Pickup} and \texttt{Move} will be successful if they have the correct identifier
and that their instruction type matches. \tabularnewline
\bottomrule
\end{longtable}

\begin{longtable}[c]{@{}p{0.2\textwidth}p{0.7\textwidth}@{}}
\toprule
& Adding Instructions to the Brain \tabularnewline
\midrule
Prerequisites & - \tabularnewline
Description & Different instructions shall be added to the Brain and
\texttt{getInstructionGraph()} will be tested to ensure that the latest added \texttt{Instruction} is always returned. \tabularnewline
Expected Result & A successful test is determined by whether the last
added instruction is returned by a call to \texttt{getInstructionGraph()} in the \texttt{Brain} class. \tabularnewline
\bottomrule
\end{longtable}

\subsubsection{Game}\label{game}

\begin{longtable}[c]{@{}p{0.2\textwidth}p{0.7\textwidth}@{}}
\toprule
& Parse an Ant Brain \tabularnewline
\midrule
Prerequisites & An ant-brain file. \tabularnewline
Description & Checks if a user's custom ant brain (expected to have
been provided in a file on disk) is valid according to the ant brain
specification. If so, constructs an in-memory programmatic
representation of the brain. \tabularnewline
Expected Result & A programmatic representation of the ant brain. It
iterates through each line of the file, tokenising the line and
validating each token. Builds a graph of ant-brain instructions which
form a finite-state machine with transitions between instructions
dependent on the conditions specified in the line. \tabularnewline
\bottomrule
\end{longtable}

\begin{longtable}[c]{@{}p{0.2\textwidth}p{0.7\textwidth}@{}}
\toprule
& Simulate 2-Player Game \tabularnewline
\midrule
Prerequisites & Two different colonies (consisting of a colour, a team
name and an ant-brain) and an ant-world. \tabularnewline
Description & a simulation of the game based on the ant finite-state
machines and their environment. \tabularnewline
Expected Result & For each turn (maximum 300,000), the ants on each cell
shall be iterated over. Each ant will perform an action based on their
ant-brain. The state of a cell (and possibly other ants) will be updated
based on the action of the ant. \tabularnewline
\bottomrule
\end{longtable}

\begin{longtable}[c]{@{}p{0.2\textwidth}p{0.7\textwidth}@{}}
\toprule
& Produce Statistics for Game \tabularnewline
\midrule
Prerequisites & The state of an ant-game at a particular turn
(accumulative) or the full history of an ant-game at the end of the
game. \tabularnewline
Description & For each team the following statistics should have been
computed: \begin{itemize}
\tightlist
\item
  Amount of food in anthill
\item
  Number of foe's ants killed
\item
  Number of team's ants left
\item
  Number of movements made
\item
  Number of markings left
\end{itemize} \tabularnewline
Expected Result & Given the state of the game at a particular turn, all
relevant statistics counters shall be updated. At the end of the game,
the statistics shall be calculated and output. \tabularnewline
\bottomrule
\end{longtable}

\begin{longtable}[c]{@{}p{0.2\textwidth}p{0.7\textwidth}@{}}
\toprule
& Simulate a Tournament \tabularnewline
\midrule
Prerequisites & A list of teams (specified by a name and an ant-brain),
and an arbitrary (but at least 1) number of ant-worlds. \tabularnewline
Description & Plays ant games between multiple users, each with their
own ant-brain specification. \tabularnewline
Expected Result & Should iterate over all unique combinations (team-1, team-2,
world) and simulate a game with that 3-tuple, with the first team
playing as red and the second team playing as black. Should keep track of all
wins, losses and statistics for each team. After all simulations, the
overall winner should be returned. \tabularnewline
\bottomrule
\end{longtable}

\begin{longtable}[c]{@{}p{0.2\textwidth}p{0.7\textwidth}@{}}
\toprule
& Uploading Ant Brains and Worlds \tabularnewline
\midrule
Prerequisites & Ant brain and ant world files. \tabularnewline
Description & The user can upload brains into an ant world with two
ant colonies, one for each player and it should check if an ant-brain
and ant world supplied by a player is syntactically well-formed. \tabularnewline
Expected Result & The program should be able to load the ant-brain and
generate random but well-formed ant worlds.\tabularnewline
\bottomrule
\end{longtable}

\begin{longtable}[c]{@{}p{0.2\textwidth}p{0.7\textwidth}@{}}
\toprule
& Two-Player Game \tabularnewline
\midrule
Prerequisites & - \tabularnewline
Description & The program should allow two players to compete in a match. \tabularnewline
Expected Result & enables two players to upload their ant-brains and
choose an ant-world, and then run the game in the ant world, taking
statistics and determining the winner of the game. \tabularnewline
\bottomrule
\end{longtable}

\begin{longtable}[c]{@{}p{0.2\textwidth}p{0.7\textwidth}@{}}
\toprule
& Tournament \tabularnewline
\midrule
Prerequisites & - \tabularnewline
Description & The program should allow users to play tournaments. \tabularnewline
Expected Result & an arbitrary number of players can upload ant-brains,
who are all paired up to play against each other. The overall tournament
winner is the ant brain that wins the most individual games. In a
tournament, each team will play every other team twice on each of the
contest worlds. Once as red and once as black. So for 23 teams, each
team will be pitted up against 22 other teams. \tabularnewline
\bottomrule
\end{longtable}

\begin{longtable}[c]{@{}p{0.2\textwidth}p{0.7\textwidth}@{}}
\toprule
& Winning a Game \tabularnewline
\midrule
Prerequisites & - \tabularnewline
Description & A test to check for the correct winner of a match. \tabularnewline
Expected Result & The match is won by the species with the most food in
its anthill at the end of 300,000 rounds.\tabularnewline
\bottomrule
\end{longtable}

\subsubsection{Ant}\label{ant}

\begin{longtable}[c]{@{}p{0.2\textwidth}p{0.7\textwidth}@{}}
\toprule
& Sensing a Cell \tabularnewline
\midrule
Prerequisites & - \tabularnewline
Description & Performs a sense operation on a particular cell of a
particular condition (cell and condition specified by the ant-brain
instruction). \tabularnewline
Expected Result & The ant's current instruction stores information about
the direction to sense in and the condition to sense for. The direction
is one of here, ahead, left-ahead or right-ahead. The ant queries the
world to determine the cell in the relevant direction, and checks the
condition on that cell. If the condition holds for that cell, it
transitions to the success state. If not, it transitions to the
fail-state. \tabularnewline
\bottomrule
\end{longtable}

\begin{longtable}[c]{@{}p{0.2\textwidth}p{0.7\textwidth}@{}}
\toprule
& Marking a Cell \tabularnewline
\midrule
Prerequisites & - \tabularnewline
Description & Performs a mark operation on the current cell, placing
a chemical marker as specified by the current ant-brain instruction. \tabularnewline
Expected Result & Tells the world to mark the ant's current cell with a
particular integer. Transitions to the next state. \tabularnewline
\bottomrule
\end{longtable}

\begin{longtable}[c]{@{}p{0.2\textwidth}p{0.7\textwidth}@{}}
\toprule
& Unmarking a Cell \tabularnewline
\midrule
Prerequisites & - \tabularnewline
Description & Performs an unmark operation on the current cell,
removing a chemical marker as specified by the current ant-brain
instruction. \tabularnewline
Expected Result & Tells the world to unmark the ant's current cell,
removing the marker with a particular identifier. Transitions to the
next state. \tabularnewline
\bottomrule
\end{longtable}

\begin{longtable}[c]{@{}p{0.2\textwidth}p{0.7\textwidth}@{}}
\toprule
& Picking up Food \tabularnewline
\midrule
Prerequisites & - \tabularnewline
Description & Picks up food from the current cell, transitioning to
one of two ant-brain states depending on whether the pick-up operation
was successful. \tabularnewline
Expected Result & Queries the world to see if the current cell has food.
If not, transitions to the fail-state. Otherwise sets the internal state
of the ant to has-food = true, as per requirement Ant/Misc.4. Removes a
food particle from the current cell. Transitions to the success state. \tabularnewline
\bottomrule
\end{longtable}

\begin{longtable}[c]{@{}p{0.2\textwidth}p{0.7\textwidth}@{}}
\toprule
& Dropping Food \tabularnewline
\midrule
Prerequisites & - \tabularnewline
Description & Drops food into the current cell, transitioning to the
next state as specified in the ant-brain finite state machine. \tabularnewline
Expected Result & Gets the current cell from the world context and adds a
food particle to it. Transitions to the next state. \tabularnewline
\bottomrule
\end{longtable}

\begin{longtable}[c]{@{}p{0.2\textwidth}p{0.7\textwidth}@{}}
\toprule
& Turning Direction \tabularnewline
\midrule
Prerequisites & - \tabularnewline
Description & Causes the internal direction attribute of the ant (as
specified in Ant/Misc.7) to change based on the current turn
instruction's specification. \tabularnewline
Expected Result & The direction specified by the instruction can be one
of left or right. This shall cause the direction of the ant to `rotate'
- i.e.~left will cause the ant to rotate one edge anticlockwise in its
hexagonal cell, and vice-versa for right. The ant will then transition
into the next state. \tabularnewline
\bottomrule
\end{longtable}

\begin{longtable}[c]{@{}p{0.2\textwidth}p{0.7\textwidth}@{}}
\toprule
& Moving \tabularnewline
\midrule
Prerequisites & - \tabularnewline
Description & Causes the ant to move forward based on its direction
into an adjacent cell (if possible). \tabularnewline
Expected Result & Based on the ant's current direction and the ant's
current cell, the world can be queried for the cell in front of the ant.
If the cell is free (not rocky and does not contain another ant), the
ant shall remove itself from the current cell and add itself to the
relevant adjacent cell. It shall then update its own resting attribute
to indicate to the game controller that it must now rest for 14 moves.
Then it will transition to the success state. If not free, the ant shall
transition to the fail-state. \tabularnewline
\bottomrule
\end{longtable}

\begin{longtable}[c]{@{}p{0.2\textwidth}p{0.7\textwidth}@{}}
\toprule
& Flip \tabularnewline
\midrule
Prerequisites & - \tabularnewline
Description & Causes the ant to transition into one of two states
depending on the result of a pseudo-random generated integer. \tabularnewline
Expected Result & Using the system's random number generator, the ant
shall produce a random integer between 0..(n-1), where n is the integer
specified by the flip instruction. If 0, the ant shall transition to
state-1. Otherwise, it shall transition to state-2.\tabularnewline
\bottomrule
\end{longtable}

\begin{longtable}[c]{@{}p{0.2\textwidth}p{0.7\textwidth}@{}}
\toprule
& Marking and Unmarking \tabularnewline
\midrule
Prerequisites & - \tabularnewline
Description & Ants perform the marking and and unmarking of chemical
cells functionality. \tabularnewline
Expected Result & Ants communicate or leave trails by means of chemical
markers.\tabularnewline
\bottomrule
\end{longtable}

\begin{longtable}[c]{@{}p{0.2\textwidth}p{0.7\textwidth}@{}}
\toprule
& Sensing \tabularnewline
\midrule
Prerequisites & - \tabularnewline
Description & Ants perform the sense function successfully. \tabularnewline
Expected Result & Each species of ants can sense (with limited
capabilities), but not modify, the markers of the other species.\tabularnewline
\bottomrule
\end{longtable}

\begin{longtable}[c]{@{}p{0.2\textwidth}p{0.7\textwidth}@{}}
\toprule
& Deaths \tabularnewline
\midrule
Prerequisites & - \tabularnewline
Description & Tests when ants are supposed to be killed. \tabularnewline
Expected Result & If an ant is adjacent to 5 (or 6) ants of the other
species then it dies. However, if an ant is in a cell in the very corner
of the grid or along one of the edges it can only have at most 4 ants
adjacent to it - therefore it cannot be killed.\tabularnewline
\bottomrule
\end{longtable}

\subsubsection{World}\label{world}

\begin{longtable}[c]{@{}p{0.2\textwidth}p{0.7\textwidth}@{}}
\toprule
& Parse Ant World File \tabularnewline
\midrule
Prerequisites & An ant-world file. \tabularnewline
Description & Checks if a user's custom ant world (expected to have
been provided in a file on disk) is valid according to the ant world
specification. If so, constructs an in-memory programmatic
representation of the world. \tabularnewline
Expected Result & Iterates through each line of the file, tokenising the
line and validating each token. Interpreting each token as a cell and
building a hexagonal programmatic structure containing cells with
properties specified in the file.\tabularnewline
\bottomrule
\end{longtable}

\begin{longtable}[c]{@{}p{0.2\textwidth}p{0.7\textwidth}@{}}
\toprule
& Spawning Ants \tabularnewline
\midrule
Prerequisites & An ant-world file. \tabularnewline
Description & Checks for successful spawning of ants. \tabularnewline
Expected Result & The two species of ants are placed in a random world
containing two anthills, some food sources, and several obstacles.\tabularnewline
\bottomrule
\end{longtable}

\begin{longtable}[c]{@{}p{0.2\textwidth}p{0.7\textwidth}@{}}
\toprule
& Simulating Behaviour \tabularnewline
\midrule
Prerequisites & An ant-world file, correct ant behaviour. \tabularnewline
Description & Checks that the behaviour of ants inside the world is correct. \tabularnewline
Expected Result & An ant world should simulate the behaviour of both
kinds of ants, using the ant-brains supplied by the players.\tabularnewline
\bottomrule
\end{longtable}

\begin{longtable}[c]{@{}p{0.2\textwidth}p{0.7\textwidth}@{}}
\toprule
& Cell Contents \tabularnewline
\midrule
Prerequisites & An ant-world file.\tabularnewline
Description & A cell holds an unlimited amount of food during the
course of the simulation. \tabularnewline
Expected Result & Food is always non-negative. \tabularnewline
\bottomrule
\end{longtable}

\begin{longtable}[c]{@{}p{0.2\textwidth}p{0.7\textwidth}@{}}
\toprule
& Cell Positioning \tabularnewline
\midrule
Prerequisites & A contest ant-world file.\tabularnewline
Description & Checks for validity of the ant world. \tabularnewline
Expected Result & There must be at least one empty cell between adjacent
non-food items. \tabularnewline
\bottomrule
\end{longtable}

\section{Acceptance Criteria}\label{acceptance-criteria}

\subsection{Introduction}\label{introduction}

This section describes the final validation and verification stage.
This acceptance criteria specification primarily focuses on the user and
release testing phase of the project, rather than the development
testing that is described in the test specification document.

The two main procedures to be considered in ensuring the acceptance of
the complete product will be:
\begin{itemize}
\item Validation: ensuring that all components
of the program satisfy the \emph{user} requirements.
\item Verification:
ensuring that all components of the program satisfy the
\emph{functional} and \emph{non-functional} requirements.
\end{itemize}

\subsection{Test Environment}\label{test-environment}

For this final testing stage, the test environment shall be made as
`realistic' as possible. That is, unless debugging, the tests shall be
carried out in a program environment that is similar to the environment
in which the client will use the program. These program environments are
specified in the non-functional requirements (see supported systems in
the non-functional requirements document). This will essentially cover
user and release testing, wherein the program shall be used by the team
as if they were a user.

In particular, the system should be tested on each of the three main
operating systems as identified in the non-functional requirements.

\subsection{Acceptance Criteria Specification}\label{acceptance-tests}

The acceptance tests shall involve using the program with the
client-provided data files (using various combinations of ant-brains and
ant-worlds). The system should be `stress-tested', meaning: 
\begin{itemize}
\item Non-conventional behaviour - experimenting with the GUI performing
relatively uncommon actions (e.g.~running a game without specifying an
ant-brain, specifying no teams in a tournament, etc.) to check that the
program does not fall into an inconsistent state.
\item Large tasks - for
example, testing the system with large ant brain and world files.
\end{itemize}
In addition to the above, the user requirements shall be individually
checked against the program by each member of the team. The
non-functional requirements should also be checked by each member of the
team, from a user's perspective.

\subsection{Release Tests}\label{gui}

\begin{longtable}[c]{@{}p{0.2\textwidth}p{0.7\textwidth}@{}}
\toprule
& Visualising Ant Worlds \tabularnewline
\midrule
Description & Uses a Swing GUI to show users a representation of the
ant world in a game at a particular turn.\tabularnewline
Expected Result & Loops through all cells in the game and draws them to
screen, with an indication of the type of cell and what it contains
specified by a colour (colours not yet specified, will be at the
discretion of the programmers and testers).\tabularnewline
\bottomrule
\end{longtable}

\begin{longtable}[c]{@{}p{0.2\textwidth}p{0.7\textwidth}@{}}
\toprule
& Reliability of Parsing User Files \tabularnewline
\midrule
Description & The software shall parse user files of up to 10,000,
even if the files are corrupt or not well-formed.\tabularnewline
Expected Result & The software does not crash more than 1 time in 1000. \tabularnewline
\bottomrule
\end{longtable}

\begin{longtable}[c]{@{}p{0.2\textwidth}p{0.7\textwidth}@{}}
\toprule
& Reliability of Game \tabularnewline
\midrule
Description & The software should crash no more than 1 time in 1000
when simulating a game or tournament.\tabularnewline
Expected Result & The software does not crash and visualisations of games
and tournaments should proceed with no visible defects (glitches) at
least 99 times out of 100. If the program crashes, a log containing the
stack trace of the error and any other pertinent details shall be
written to the program's working directory, titled
``ant-game-crash-{[}current unix time{]}.log''. \tabularnewline
\bottomrule
\end{longtable}

\begin{longtable}[c]{@{}p{0.2\textwidth}p{0.7\textwidth}@{}}
\toprule
& Software Usability \tabularnewline
\midrule
Description & The software should have a minimal learning curve for
new users. That is, the GUI should have intuitive and clear controls
that step the user(s) through each stage of the game without being
required to follow external instructions (not including understanding
the rules of the game and the development of a user's ant-brain file). \tabularnewline
Expected Result & The software shall present a help menu, accessible at
all times in the program, which upon clicking displays a popup user
interface containing the searchable user documentation for the program. \tabularnewline
\bottomrule
\end{longtable}

\begin{longtable}[c]{@{}p{0.2\textwidth}p{0.7\textwidth}@{}}
\toprule
& Software Performance \tabularnewline
\midrule
Description & Parsing user files (ant-brain, ant-world) should take
no longer than 5 seconds, The GUI should not hang (that is, be
unresponsive to user interaction) for more than one second after any
user interaction. The two-player game simulation should take no longer
than 30 seconds for 300,000 rounds. \tabularnewline
Expected Result & The software performs to the specified performance
timings above. \tabularnewline
\bottomrule
\end{longtable}

\begin{longtable}[c]{@{}p{0.2\textwidth}p{0.7\textwidth}@{}}
\toprule
& Operating System Compatibility \tabularnewline
\midrule
Description & The software shall be cross-platform across commonly
used personal desktop operating systems (Windows 7+, Mac OS X 10.8.3+,
Linux) that support Java 8.0. \tabularnewline
Expected Result & The software functionals correctly on all platforms.\tabularnewline
\bottomrule
\end{longtable}

\end{document}
